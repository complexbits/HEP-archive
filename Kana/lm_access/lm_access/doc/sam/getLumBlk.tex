\documentclass[12pt]{article}
\parindent 0pt
\parskip 10 pt
\begin{document}
\author{Heidi Schellman, Maciej Przybycien, Sunghwan Ahn, \\Michael Begel, Gregor Geurkov, Alex Kupco
\\John Hauptman, Sehwook Lee, Prolay Mal, Jaewon Park,\\ Rich Partridge, Marco Verzocchi \\
The Luminosity ID Group}
\title{D0 Note 3937:User Interface to Find the Luminosity Blocks Corresponding to Data Files}
\date{\today}
\maketitle

\section{Introduction}
In order to find the luminosity for any data sample, you need to know the time period
over which the data were taken and the luminosity corresponding to that time period.
Every minute, we log the luminosity information from the luminosity monitors.
Each such logging cycle is called a luminosity block (LBN).  Calculations of
luminosity then become a two step process. First for any given data sample, you
need to find the set of luminosity blocks corresponding to
that dataset,  then given that set of luminosity blocks you
can access the full luminosity information, find the bad blocks, make
corrections and derive
with a final luminosity estimate for your sample.\footnote{You will of course need
to run through your sample again and eliminate any events that occurred in bad
luminosity blocks.}





This note describes the method currently being used to determine the set of luminosity
blocks for a given data sample.  It does not include the case of exclusive streaming
as that is not being used at present.

\section{SAM file storage}

The online data logger closes and opens files at LBN boundaries.  A given LBN
should appear in one and only one raw data file in any given data stream.  For each raw
data file, the LBN range is stored in sam. If raw file is in your data sample,
the LBN's  in its range will count towards your final luminosity. \footnote{If a given
luminosity block contains no triggered events, even if the detector was live, it
will not be included in the file and that LBN will not be in the list.  This is
one of the corrections that needs to be made in order to get a full luminosity
for a data sample.}% Michael made me put this in.}

When we processes a file through reconstruction or offline filtering, it is written
back into SAM with metadata which describe the processing step.  The important information
for luminosity is the parentage of the derived file - which input files created the
output file?  If one knows the parentage for a file, you can trace it back to the
original raw data file, and get the LBN ranges for all the raw files.  That
set of LBN's is the luminosity set for that data sample.

\subsection{Paternity}

For the parentage method to work, all files in the processing chain must be
included, even those for which no events showed up in the final sample.

WARNING: Derived samples not made as part of official production may not
have complete parentage and should be used with extreme caution unless the
production chain has been documented.

The rest of this section describes methods for avoiding common pitfalls in
file processing.


The root files produced by official reconstruction are believed to
have complete parentage information.  This is assured by the requirement
that the reconstruction job return success before the root file
is generated and that  all of the parent reconstructed files be stored into sam before the
root file can be stored.

\subsubsection{Split files}
However reconstructed files may have
been split because they became too large. These files are rare but they
do not have correct parentage information by themselves.
Because the file had to be split at an arbitrary point in the middle of
an LBN, each resulting file has the full parentage for the parent file
stored with it.  If an analysis processes all of the reconstructed files for a given
input file, (the usual situation) the parentage information is complete, but if one of the
files is missing, the parentage may include additional LBN's from
the missing file.

\subsubsection{Filtering followed by merging}
Another potential source of errors is event filtering followed by file merging.  If, after
filtering, a file ends up with zero events, the parentage information will not
get stored with the merged file and the luminosity block ranges for the merged file
will be incomplete.
Currently, we handle this case by  always
writing one event to any output file when filtering.  This is done calling the \tt special \rm
package in d0reco and putting \tt firstevent.stream \rm into the
list of event tags in \tt WriteEvent.\rm  If that output file is then merged
with others, it will bring its parents with it and retain the full luminosity
for the sample.

\section{The \tt lm\_access \rm interface}

The \tt lm\_access/scripts \rm package in CVS is designed to do
that tracing for you.  It contains 2 scripts.


\begin{verbatim}
getLumBlk.py <source-options> [--cache=<cache-location>]
\end{verbatim}

Source options are:

\begin{verbatim}
--file=<filename> does a single file
--list=<filename> does all the files listed in a file
--defname=<definition>  does all the files in a sam dataset definition
\end{verbatim}

The \tt--cache\rm  argument points to an optional cache file where previous results of
such queries are stored. Tracing the parentage of the files is very time consuming
as it involves multiple nested calls to the database, so caching the results is very useful.

getFileParentsRecursive.py is used by the other scripts to trace the parentage.  It is slow.


Output of the program looks like:

\begin{verbatim}
reco_recoS_all_0000139615_mrg_005-005.raw_p10.11.00_p10.13.00-a_000\
 filtered-reco 2411 0 1 < 139615 : 738565 - 738570 >
\end{verbatim}

Where the format is:
\begin{description}
\item{\bf Filename}
\item{\bf Data Tier}
\item{\bf Number of raw events in all parent files}
\item{\bf Status:} default = 0, split file = 1
\item{\bf Number of events in this file}
\item{\bf List of LBN ranges} in format \tt RUNNUM : LUMMIN - LUMMAX \rm\break
All are enclosed in triangular brackets.
\end{description}

From these files, you can find the set of luminosity ranges for any derived file.



\section{How it works}
Under the covers, these programs use the file information scripts which can be
accessed from both python and the command line.

An example of use of the command line interface, which dumps everything.



\begin{verbatim}
sam dump file --filename=myfile

attributeList: runNumber: 139622, physicalDatastreamName: all,
logicalDatastreamName: generic, dataTier: filtered-reco, fileStatus: available,
 formatInfo: , minbiasNumber: , minbiasType: , eventCount: 2,
 kbyteFileSize: 685, firstEventNumber: 75210, lastEventNumber: 78481,
 lumMin: -1, lumMax: -1, filePartition: -1, createDate: 01/14/2002 12:47:40,
 createUser: samdbs, startTime: 01/14/2002 10:06:01, endTime: ,
 family: reconstruction, version: p10.13.00-a, applName: d0reco,
 projectName: farm.p10.13.00-a.10797, projSnapId: 14525,
 projectDefName: farm-recos-after-139000-p10.13.00-a_20020114095221
children list:  ['recoA_reco_recoS_all_mrg_0000139615-005_0000139633-007.\
raw_p10.11.00_p10.13.00-a.root']
parent list:  ['recoS_all_0000139622_mrg_007-008.raw_p10.11.00']
\end{verbatim}

An example using the Python interface:
\begin{verbatim}
....
import file_client

def getFileParents(filename):


        try:
            fileinfo = file_client.APIGetFileInfo(filename)
        except SAMDbServer.DBError, oerr:
            print "**** Terminating execution due to database error"
            return ["DBError"]
        except SAMDbServer.DataFileNotFound:
            print "File not found"
            return ["NoSuchFile"]

        return fileinfo.parentList



\end{verbatim}
\section {Cache Locations}

Caches for different production runs are stored in CVS in package \tt lm\_access/cache \rm.
The naming convention is \tt$<$data\_tier$>$\_$<$production-version$>$.cache \rm

They are merged into a single file in directory \tt d0mino:/prj\_root/846/cd\_1/lm\_access\_cache/cache\_v1 \rm

\end{document}
